\documentclass[10pt]{article}
\usepackage[usenames]{color} %フォントカラー
\usepackage{amssymb} %数式記号
\usepackage{amsmath} %数式
\usepackage[utf8]{inputenc} %発音区別符アルファベットの直接入力
\begin{document}
\begin{align*}\dot{m}
=
\frac{1}{2} \left[ \rho_{L} \left( U_{nL}+ | \overline{ U_{n}} |^{+} \right) + \rho_{R} \left( U_{nR} - | \overline{U_{n}} |^{-} \right) - \frac{\chi}{\overline{c} }  \left( p_{R}- p_{L} \right) \right]\end{align*}
\end{document}